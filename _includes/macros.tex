%------------------------------------------------------------------------- arrows


\newcommand{\raw}{\rightarrow}
\newcommand{\xraw}{\xrightarrow}
\newcommand{\Raw}{\Rightarrow}
\renewcommand{\dot}{\centerdot}
\newcommand{\pr}{^\prime}
\newcommand{\dar}{\downarrow}

\newcommand{\cd}[2][]{\vcenter{\hbox{\xymatrix#1{#2}}}}
\newcommand{\pullback}[1][dr]{\save*!/#1-1.2pc/#1:(-1,1)@^{|-}\restore}

\newcommand{\op}[1]{#1^{\mathrm{op}}}
\newcommand{\Hom}{\textsl{Hom}}
\newcommand{\oo}{\circ}
\newcommand{\intvl}{\mathbb{I}}



\usetikzlibrary{decorations.pathmorphing,shapes}
\newcounter{sarrow}
\newcommand\xrsquigarrow[1]{%
\stepcounter{sarrow}%
\begin{tikzpicture}[decoration=snake]
\node (\thesarrow) {\strut#1};
\draw[->,decorate] (\thesarrow.north west) -- (\thesarrow.south west);
\end{tikzpicture}%
}




%-------------------------------------------------------------

\makeatletter
\newdimen\itex@wd%
\newdimen\itex@dp%
\newdimen\itex@thd%
\def\itexspace#1#2#3{\itex@wd=#3em%
\itex@wd=0.1\itex@wd%
\itex@dp=#2ex%
\itex@dp=0.1\itex@dp%
\itex@thd=#1ex%
\itex@thd=0.1\itex@thd%
\advance\itex@thd\the\itex@dp%
\makebox[\the\itex@wd]{\rule[-\the\itex@dp]{0cm}{\the\itex@thd}}}
\makeatother






%-------------------------------------------------------------------------- spaces and categories


\newcommand{\Setbf}{\mbox{\textbf{Set}}}
\newcommand{\Set}{\mbox{\textit{Set}}}
\def\Sett{\mathop{\mathcal{S}\! {\it et}}\nolimits}

\newcommand{\Top}{\mbox{\textit{Top}}}
\def\Topp{\mathop{\mathcal{T}\! {\it op}}\nolimits}
\def\Topps{\mathop{\mathcal{T}\! {\it op_*}}\nolimits}
\def\hTop{\box{\textit{hTop}}}
\def\hTopp{\mathop{h\mathcal{T}\! {\it op}}\nolimits}
\def\hTopps{\mathop{h\mathcal{T}\! {\it op_*}}\nolimits}


\newcommand{\Topos}{{\mathfrak{Top}}}
\newcommand{\BTopos}{{\mathfrak{BTop}}}

\newcommand{\Loc}{\mbox{\textit{Loc}}}
\def\Loc{\mathop{\mathcal{L}\! {\it oc}}\nolimits}
\newcommand{\SP}{{\mathcal S}p}

\newcommand{\Grp}{\mbox{\textit{Grp}}}
\def\Grpp{\mathop{\mathcal{G}\! {\it rp}}\nolimits}
\newcommand{\AbGrp}{\mbox{\textit{AbGrp}}}
\newcommand{\Ab}{\mbox{\textit{AbGrp}}}
\def\AbGrpp{\mathop{\mathcal{A}\!{\it b}}\nolimits}

\newcommand{\Span}{\mbox{\textit{Span}}}
\newcommand{\Fib}{\mbox{\textit{Fib}}}
\newcommand{\OpFib}{\mbox{\textit{OpFib}}}
\newcommand{\DFib}{\mbox{\textit{DFib}}}
\newcommand{\DoFib}{\mbox{\textit{DoFib}}}
\newcommand{\Ring}{\mbox{\textit{Ring}}}
\newcommand{\CGHaus}{\mbox{\textit{CGHaus}}}




\def\Map{\mathop{\rm Map}\nolimits}
\def\Haus{\mathop{\mathcal{H}\! {\it AUS}}\nolimits}
\def\CGHaus{\mathop{\mathcal{CG}\! {\it HAUS}}\nolimits}
\def\sAbGrp{\mathop{s\mathcal{A}\!{\it b}}\nolimits}

\def\MAP{\mathop{\mathcal{M}\! {\it ap}}\nolimits}
\newcommand{\ye}{\mbox{\textit{y}}}
\newcommand{\yon}{\mbox{\textit{y}}}
\newcommand{\Psh}{\mbox{\textit{Psh}}}
\newcommand{\Sh}{\mbox{\textit{Sh}}}

%\newcommand{\Hom}{\mbox{\textbf{Hom}}}
\newcommand{\Con}{{\mathfrak{Con}}}
\newcommand{\Pos}{{\mathfrak{Poset}}}
\newcommand{\thT}{\mathbb{T}}   
\newcommand{\Mod}{\mathbf{Mod}}
\newcommand{\Th}{\mathbf{Th}}
\newcommand{\Cl}{\mathbf{Cl}}
\newcommand{\ob}{\mathbf{Ob}}
\newcommand{\bun}{\mathbf{Bun}} 
\newcommand{\Geom}{\mathbf{Geom}}





\newcommand{\n}{\mathbf{n}}
\newcommand{\m}{\mathbf{m}}
\newcommand{\zero}{\mathbf{0}}
\newcommand{\one}{\mathbf{1}}
\newcommand{\two}{\mathbf{2}}
\newcommand{\three}{\mathbf{3}}
\newcommand{\four}{\mathbf{4}}




\def\cat{\mathop{\mathcal{C}\! {\it at}}\nolimits}
\def\catt{\mathop{\mathcal{C}\! {\it AT}}\nolimits}
\def\Cat{\mathop{\mathfrak{Cat}}}
\def\Catt{\mathop{\mathfrak{CAT}}}
\def\CAT{\mathop{\mathbf{2} \mathfrak{Cat}}}
\def\CATT{\mathop{\mathbf{2} \mathfrak{CAT}}}








\newcommand{\CA}{{\mathcal A}}
\newcommand{\CC}{{\mathcal C}}
\newcommand{\CB}{{\mathcal B}}
\newcommand{\CD}{{\mathcal D}}
\newcommand{\CE}{{\mathcal E}}
\newcommand{\CF}{{\mathcal F}}
\newcommand{\CM}{{\mathcal M}}
\newcommand{\CI}{{\mathcal I}}
\newcommand{\CJ}{{\mathcal J}}
\newcommand{\CK}{{\mathcal K}}
\newcommand{\CL}{{\mathcal L}}
\newcommand{\CR}{{\mathcal R}}
\newcommand{\CS}{{\mathcal S}}
\newcommand{\CW}{{\mathcal W}}
\newcommand{\CP}{{\mathcal P}}
\newcommand{\CU}{{\mathcal U}}
\newcommand{\CV}{{\mathcal V}}
\newcommand{\CX}{{\mathcal X}}
\newcommand{\CY}{{\mathcal Y}}
\newcommand{\CO}{{\mathcal O}}













%------------------------------------------------------------------------- Imported from Shulman's framed bicats




\hyphenation{para-met-riz-ed}
\hyphenation{pseudo-functor}
\hyphenation{pseudo-functors}
\hyphenation{pseudo-functoriality}

% 2-categories and bicategories
\newcommand{\MonCat}{\ensuremath{\mathcal{M}\mathit{on}\mathcal{C}\mathit{at}}}
\newcommand{\MonCatl}{\ensuremath{\mathcal{M}\mathit{on}\mathcal{C}\mathit{at}_{\ell}}}
\newcommand{\BrMonCat}{\ensuremath{\mathcal{B}\mathit{r}\mathcal{M}\mathit{on}\mathcal{C}\mathit{at}}}
\newcommand{\SymMonCat}{\ensuremath{\mathcal{S}\mathit{ym}\mathcal{M}\mathit{on}\mathcal{C}\mathit{at}}}

\newcommand{\FibB}{\ensuremath{\mathcal{F}\mathit{ib}_{\sB}}}
\newcommand{\Fibc}{\ensuremath{\mathcal{F}\mathit{ib}_{\mathit{op}\ell}}}
\newcommand{\FibcB}{\ensuremath{\mathcal{F}\mathit{ib}_{\mathit{op}\ell,\sB}}}
\newcommand{\calMod}{\ensuremath{\mathcal{M}\mathit{od}}}
\newcommand{\calMat}{\ensuremath{\mathcal{M}\mathit{at}}}
\newcommand{\calDist}{\ensuremath{\mathcal{D}\mathit{ist}}}
\newcommand{\calEx}{\ensuremath{\mathcal{E}\mathit{x}}}
\newcommand{\Cart}{\ensuremath{\mathcal{C}\mathit{art}}}
\newcommand{\biCartpsc}{\ensuremath{\mathit{bi}\mathcal{C}\mathit{art}^{\mathrm{psc}}}}

% Tricategories, etc.
\newcommand{\Bicatc}{\ensuremath{\mathfrak{Bicat}_{\mathit{op}\ell}}}

% related 1-categories 
\newcommand{\bfBicat}{\ensuremath{\mathbf{Bicat}}}

% 2-categories of double categories
\newcommand{\Dbl}{\ensuremath{\mathcal{D}\mathit{bl}}}
\newcommand{\Dbll}{\ensuremath{\mathcal{D}\mathit{bl}_{\ell}}}
\newcommand{\Dblc}{\ensuremath{\mathcal{D}\mathit{bl}_{\mathit{op}\ell}}}
\newcommand{\FrBi}{\ensuremath{\mathcal{F}\mathit{r}\mathcal{B}\mathit{i}}}
\newcommand{\FrBil}{\ensuremath{\mathcal{F}\mathit{r}\mathcal{B}\mathit{i}_{\ell}}}
\newcommand{\FrBilq}{\ensuremath{\mathcal{F}\mathit{r}\mathcal{B}\mathit{i}_{\ell}^q}}
\newcommand{\FrBilnq}{\ensuremath{\mathcal{F}\mathit{r}\mathcal{B}\mathit{i}_{\ell,n}^q}}
\newcommand{\FrBiq}{\ensuremath{\mathcal{F}\mathit{r}\mathcal{B}\mathit{i}^q}}
\newcommand{\FrBic}{\ensuremath{\mathcal{F}\mathit{r}\mathcal{B}\mathit{i}_{\mathit{op}\ell}}}
\newcommand{\MF}{\ensuremath{\mathcal{MF}}} % monoidal fibrations
\newcommand{\MFB}{\ensuremath{\mathcal{MF}_\sB}} % monoidal fibrations
\newcommand{\MFc}{\ensuremath{\mathcal{MF}_{\mathit{op}\ell}}}
\newcommand{\MFl}{\ensuremath{\mathcal{MF}_\ell}}
\newcommand{\MbF}{\ensuremath{\mathcal{M}\mathit{bi}\mathcal{F}}} % monoidal bifibrations
\newcommand{\MbFl}{\ensuremath{\mathcal{M}\mathit{bi}\mathcal{F}_\ell}} % monoidal bifibrations
\newcommand{\BMF}{\ensuremath{\mathcal{BMF}}} % braided monoidal fibrations
\newcommand{\SMF}{\ensuremath{\mathcal{SMF}}} % symmetric monoidal fibrations
\newcommand{\BMFB}{\ensuremath{\mathcal{BMF}_\sB}} % braided monoidal fibrations
\newcommand{\SMFB}{\ensuremath{\mathcal{SMF}_\sB}} % symmetric monoidal fibrations
\newcommand{\MFfr}{\ensuremath{\MF^{\mathrm{fr}}}} % frameable monoidal fibrations
\newcommand{\MFfrc}{\ensuremath{\MFc^{\mathrm{fr}}}}
\newcommand{\MFfrl}{\ensuremath{\MFl^{\mathrm{fr}}}}

% double categories

\newcommand{\CMod}{\ensuremath{\mathbb{CM}\mathbf{od}}}
\newcommand{\Comod}{\ensuremath{\mathbb{C}\mathbf{omod}}}

\newcommand{\Dist}{\ensuremath{\mathbb{D}\mathbf{ist}}}
\newcommand{\Mat}{\ensuremath{\mathbb{M}\mathbf{at}}}
\newcommand{\nCob}{\ensuremath{n\mathbb{C}\mathbf{ob}}}
\newcommand{\Ex}{\ensuremath{\mathbb{E}\mathbf{x}}}
\newcommand{\Sp}{\ensuremath{\mathbb{S}\mathbf{p}}}
\newcommand{\Adj}{\ensuremath{\mathbb{A}\mathbf{dj}}}

% framing 2-functor
\newcommand{\bbFr}{\ensuremath{\mathbb{F}\mathbf{r}}}

% monoidal fibrations
\newcommand{\ttFam}{\ensuremath{\mathtt{Fam}}}
\newcommand{\ttMod}{\ensuremath{\mathtt{Mod}}}
\newcommand{\ttCMod}{\ensuremath{\mathtt{CMod}}}
\newcommand{\ttArr}{\ensuremath{\mathtt{Arr}}}
\newcommand{\ttRetr}{\ensuremath{\mathtt{Retr}}}
\newcommand{\ttSp}{\ensuremath{\mathtt{Sp}}}
\newcommand{\ttF}{\ensuremath{\mathtt{F}}}
\newcommand{\ttC}{\ensuremath{\mathtt{C}}}

% generic double categories
\newcommand{\bbDz}{{\ensuremath{\mathbb{D}_0}}}
\newcommand{\bbDo}{{\ensuremath{\mathbb{D}_1}}}
\newcommand{\bbEz}{{\ensuremath{\mathbb{E}_0}}}
\newcommand{\bbPz}{{\ensuremath{\mathbb{P}_0}}}
\newcommand{\bbPo}{{\ensuremath{\mathbb{P}_1}}}



% Calligraphic letters
\newcommand{\calA}{\ensuremath{\mathcal{A}}}
\newcommand{\calB}{\ensuremath{\mathcal{B}}}
\newcommand{\calD}{\ensuremath{\mathcal{D}}}
\newcommand{\calE}{\ensuremath{\mathcal{E}}}
\newcommand{\calF}{\ensuremath{\mathcal{F}}}
\newcommand{\calK}{\ensuremath{\mathcal{K}}}
\newcommand{\calM}{\ensuremath{\mathcal{M}}}

% blackboard bold letters
\newcommand{\bbD}{\ensuremath{\mathbb{D}}}
\newcommand{\bbE}{\ensuremath{\mathbb{E}}}
\newcommand{\bbF}{\ensuremath{\mathbb{F}}}
\newcommand{\bbP}{\ensuremath{\mathbb{P}}}
\newcommand{\bbZ}{\ensuremath{\mathbb{Z}}}
\newcommand{\bbJ}{\ensuremath{\mathbb{J}}}
% fraktur letters
\newcommand{\fa}{\ensuremath{\mathfrak{a}}}
\newcommand{\fl}{\ensuremath{\mathfrak{l}}}
\newcommand{\fp}{\ensuremath{\mathfrak{p}}}
\newcommand{\fq}{\ensuremath{\mathfrak{q}}}
\newcommand{\fr}{\ensuremath{\mathfrak{r}}}
\newcommand{\fs}{\ensuremath{\mathfrak{s}}}
\newcommand{\ft}{\ensuremath{\mathfrak{t}}}
\newcommand{\fu}{\ensuremath{\mathfrak{u}}}
\newcommand{\fx}{\ensuremath{\mathfrak{x}}}
\newcommand{\fm}{\ensuremath{\mathfrak{m}}}

% fraktur with tilde
\newcommand{\fatil}{\ensuremath{\widetilde{\mathfrak{a}}}}

% bars
\newcommand{\fbar}{\ensuremath{\overline{f}}}
\newcommand{\gbar}{\ensuremath{\overline{g}}}
\newcommand{\kbar}{\ensuremath{\overline{k}}}

% tildes
\newcommand{\ftil}{\ensuremath{\widetilde{f}}}
\newcommand{\gtil}{\ensuremath{\widetilde{g}}}

% greek with tildes
\newcommand{\altil}{\ensuremath{\widetilde{\alpha}}}
\newcommand{\bltil}{\ensuremath{\widetilde{\beta}}}

% greek letters
\newcommand{\ep}{\ensuremath{\varepsilon}}

\newcommand{\ph}{\ensuremath{\varphi}}

% some categories

\newcommand{\Fam}{\ensuremath{\mathbf{Fam}}}



% some operators
\newcommand{\Id}{\ensuremath{\operatorname{Id}}}
\newcommand{\Ho}{\ensuremath{\operatorname{Ho}}}
\newcommand{\ad}{\ensuremath{\operatorname{ad}}}
\newcommand{\Ad}{\ensuremath{\operatorname{Ad}}}
\newcommand{\Sym}{\ensuremath{\operatorname{Sym}}}
\newcommand{\Pull}{\ensuremath{\operatorname{Pull}}}
\newcommand{\Push}{\ensuremath{\operatorname{Push}}}
\newcommand{\dom}{\ensuremath{\operatorname{dom}}}
\newcommand{\cod}{\ensuremath{\operatorname{cod}}}


% arrows and categorical symbols
\newcommand{\dn}{\downarrow}
%\newcommand{\op}{\ensuremath{^{\mathit{op}}}}
\newcommand{\co}{\ensuremath{^{\mathit{co}}}}
\newcommand{\adj}{\dashv}
\newdir{ >}{{}*!/-10pt/@{>}}
\newdir{ |}{{}*!/-10pt/@{|}}
\newdir{> }{!/10pt/@{>}*{}}
\newcommand{\iso}{\cong}
\newcommand{\eqv}{\simeq}
\newcommand{\too}[1][]{\ensuremath{\overset{#1}{\longrightarrow}}}
\newcommand{\oot}[1][]{\ensuremath{\overset{#1}{\longleftarrow}}}
%\renewcommand{\to}[1][]{\ensuremath{\overset{#1}{\rightarrow}}}
\renewcommand{\to}{\ensuremath{\rightarrow}}
\newcommand{\toto}{\ensuremath{\rightrightarrows}}
\newcommand{\toot}{\ensuremath{\rightleftarrows}}
\newcommand{\into}{\ensuremath{\hookrightarrow}}
\newcommand{\hto}{\ensuremath{\,\mathaccent\shortmid\rightarrow\,}}
\newcommand{\vto}{\to}
\newcommand{\sto}[2]{\ensuremath{\underset{#1}{\overset{#2}{\Longrightarrow}}}}
\newcommand{\hop}{\ensuremath{^{\mathit{h\cdot{}op}}}}
\newcommand{\maps}{\colon}
\newcommand{\spam}{\,:\!}

% miscellaneous symbols
\newcommand{\del}{\ensuremath{\partial}}
\newcommand{\sm}{\wedge}
\newcommand{\exsm}{\ensuremath{\barwedge}}%{\,\ensuremath{\overline{\wedge}}\,}
\newcommand{\ten}{\ensuremath{\otimes}}
\newcommand{\xrhd}{\;\overline{\rhd}\;}
\newcommand{\xlhd}{\;\overline{\lhd}\;}








%---------------------------------------------------------------

\newcommand{\pullbackcorner}[1][dr]{\save*!/#1+1.2pc/#1:(1,-1)@^{|-}\restore}
\newcommand{\pushoutcorner}[1][dr]{\save*!/#1-1.2pc/#1:(-1,1)@^{|-}\restore}
\newcommand{\pullbackcornerr}[1][dr]{\save*!/#1+1.3pc/#1:(1,-1)@^{|-}\restore}
\newcommand{\pushoutcornerr}[1][dr]{\save*!/#1-1.3pc/#1:(-1,1)@^{|-}\restore}

%----------------------------------------------------------



%% Fix array
\newcommand{\itexarray}[1]{\begin{matrix}#1\end{matrix}}
%% \itexnum is a noop
\newcommand{\itexnum}[1]{#1}
%-----------------------------------------------------------

\newcommand{\underoverset}[3]{\underset{#1}{\overset{#2}{#3}}}
\newcommand{\widevec}{\overrightarrow}
\newcommand{\nearr}{\nearrow}
\newcommand{\nwarr}{\nwarrow}
\newcommand{\searr}{\searrow}
\newcommand{\swarr}{\swarrow}
\newcommand{\curvearrowbotright}{\curvearrowright}
\newcommand{\uparr}{\uparrow}
\newcommand{\downuparrow}{\updownarrow}
\newcommand{\duparr}{\updownarrow}
\newcommand{\updarr}{\updownarrow}
\newcommand{\gt}{>}
\newcommand{\lt}{<}
\newcommand{\map}{\mapsto}
\newcommand{\embedsin}{\hookrightarrow}
%-------------------------------------------------
\newcommand{\lang}{\langle}
\newcommand{\rang}{\rangle}
\newcommand{\Union}{\bigcup}
\newcommand{\Intersection}{\bigcap}
\newcommand{\Oplus}{\bigoplus}
\newcommand{\Otimes}{\bigotimes}
\newcommand{\Wedge}{\bigwedge}
\newcommand{\Vee}{\bigvee}
\newcommand{\coproduct}{\coprod}
\newcommand{\product}{\prod}
\newcommand{\closure}{\overline}
\newcommand{\integral}{\int}
\newcommand{\doubleintegral}{\iint}
\newcommand{\tripleintegral}{\iiint}
\newcommand{\quadrupleintegral}{\iiiint}
\newcommand{\conint}{\oint}
\newcommand{\contourintegral}{\oint}
\newcommand{\infinity}{\infty}
\newcommand{\bottom}{\bot}
\newcommand{\minusb}{\boxminus}
\newcommand{\plusb}{\boxplus}
\newcommand{\timesb}{\boxtimes}
\newcommand{\intersection}{\cap}
\newcommand{\union}{\cup}
\newcommand{\Del}{\nabla}
\newcommand{\odash}{\circleddash}
\newcommand{\negspace}{\!}
\newcommand{\widebar}{\overline}
\newcommand{\textsize}{\normalsize}
\renewcommand{\scriptsize}{\scriptstyle}
\newcommand{\scriptscriptsize}{\scriptscriptstyle}
\newcommand{\mathfr}{\mathfrak}

%-----------------------------------------------------------------
% Theorem Environments


\newtheoremstyle{mystyle}
  {\topsep}
  {\topsep}
  {}
  {}
  {\scshape}
  {.}
  {.5em}
  {}
  
  
\theoremstyle{mystyle}
\newtheorem{defn}{Definition}[section]
\newtheorem{lem}[defn]{Lemma}
\newtheorem{pro}[defn]{Proposition}
\newtheorem{thm}[defn]{Theorem}


\newtheorem{conj}[defn]{Conjecture}
\newtheorem{exmp}[defn]{Example}


\theoremstyle{remark}
\newtheorem*{rem}{Remark}
\newtheorem*{cor}{Corollary}



%------------------------------------------------------------------
\newcommand{\clrr}{\tikz[baseline]\node[beameralert=2,anchor=base]}
\newcommand{\clr}{\tikz[baseline]\node[beameralert=1,anchor=base]}
\newcommand\restr[2]{{% we make the whole thing an ordinary symbol
  \left.\kern-\nulldelimiterspace % automatically resize the bar with \right
  #1 % the function
  \vphantom{\big|} % pretend it's a little taller at normal size
  \right|_{#2} % this is the delimiter
  }}



%------------------------------------------------------------------ adjunction

%------------------------------------------------------------imported from Steve's paper SK 4 AU 



\newcommand{\BC}{\mathbf{C}}
\newcommand{\AU}{\mathbf{AU}}
\newcommand{\AUpres}[1]{\AU\langle #1 \rangle}
\newcommand{\thob}{\mathbb{O}}  
\newcommand{\skext}{\subset}                 % sketch extension
\newcommand{\Sk}{\mathsf{Sk}}                % graph or cat of sketches




\newcommand{\skeqext}{\Subset}               % sketch equivalence extension
\newcommand{\skhom}{\lessdot}              % sketch homomorphism
\newcommand{\skhommap}{\gtrdot}               % sketch homomorphism backwards
\newcommand{\thextmap}{\supset}              % theory extension map
\newcommand{\theqextmap}{\Supset}            % theory equivalence extension map
\newcommand{\thmorphmap}{\gtrdot}            % theory morphism map
\newcommand{\skdiag}{\sim}                   % diagrams in sketches






\newcommand{\skn}{\mathrm{G}^{0}}    % nodes
\newcommand{\sknid}{\mathrm{s}}       % identity

\newcommand{\ske}{\mathrm{G}^{1}}    % edges
\newcommand{\skface}{\mathrm{d}}     % faces (for edges and triangles)
\newcommand{\skedom}{\skface_0}     % domain
\newcommand{\skecod}{\skface_1}     % codomain

%\newcommand{\skid}{\mathrm{D}^{0}}   % diagrams for identities
%\newcommand{\skide}{\mathrm{e}}      % morphism

\newcommand{\sktri}{\mathrm{G}^{2}}  % diagrams for binary composites
% ??? Are these numbers right?
\newcommand{\sktril}{\skface_0}   % first
\newcommand{\sktrir}{\skface_2}   % second
\newcommand{\sktric}{\skface_1}   % composite

\newcommand{\skut}{\mathrm{U}^{\qeqt}} % universals for terminals
\newcommand{\skutn}{\mathrm{t}}      % terminal node

\newcommand{\skupb}{\mathrm{U}^{\mathrm{pb}}} % universals for pullbacks
\newcommand{\skupbtri}{\Gamma}       % triangle for half of pb square

\newcommand{\skui}{\mathrm{U}^{\qeqi}} % universals for initials
\newcommand{\skuin}{\mathrm{i}}      % initial node

\newcommand{\skupo}{\mathrm{U}^{\mathrm{po}}} % universals for pushouts
\newcommand{\skupotri}{\Gamma'}       % triangle for half of po square

\newcommand{\skul}{\mathrm{U}^{\mathrm{list}}} % universals for list objects L = List(A)
\newcommand{\skulpb}{\Lambda_2}      % universal for LxA
\newcommand{\skult}{\Lambda_0}       % universal for terminal
\newcommand{\skule}{\mathrm{e}}      % empty list
\newcommand{\skulcons}{\mathrm{c}}   % cons




\newcommand{\qeqobj}{\mathrm{obj}}          % sort for objects
\newcommand{\qeqarr}{\mathrm{arr}}          % sort for morphisms
\newcommand{\qeqdom}{\mathsf{d}}            % domain
\newcommand{\qeqcod}{\mathsf{c}}            % codomain
\newcommand{\qeqid}{\mathsf{id}}            % identity morphism
\newcommand{\qeqcomp}{\circ}                % composition, binary, applicational order
\newcommand{\qeqt}{1}                 % terminal object
\newcommand{\qeqtfill}{\mathop{!}\nolimits^{\qeqt}}                 % unique morphism to terminal object
\newcommand{\qeqprodproj}{\mathsf{p}}       % projections from product
\newcommand{\qeqproj}[2]{\qeqprodproj_{#1,#2}}% projections from pullback
\newcommand{\qeqpb}[2]{\mathsf{P}_{#1,#2}}             % pullback object
\newcommand{\qeqpbfill}[4]{\left\langle #1, #2 \right\rangle_{#3,#4}} % pullback fillin
\newcommand{\qeqprodfill}[2]{\left\langle #1, #2 \right\rangle} % product fillin
\newcommand{\qeqeq}[2]{\mathsf{eq}_{#1,#2}}  % equalizer morphism for #1,#2
\newcommand{\qeqeqdom}[2]{\mathsf{E}_{#1,#2}}% equalizer object for #1,#2
%\newcommand{\qeqpbu}{\mathsf{u}}            % uniqueness for pullback fillin

\newcommand{\qeqi}{0}              % initial object
\newcommand{\qeqifill}{\mathop{!}\nolimits^{\qeqi}}                 % unique morphism from initial object
\newcommand{\qeqinj}[2]{\mathsf{q}_{#1,#2}} % injections to pushout
\newcommand{\qeqpo}[2]{\mathsf{Q}_{#1,#2}}             % pushout object
\newcommand{\qeqpofill}[4]{\left[ #1, #2 \right]_{#3,#4}} % pushout fillin
%\newcommand{\qeqpou}{\mathsf{v}}            % uniqueness for pushout fillin
\newcommand{\qequc}{\mathsf{uc}}            % unique choice (balance)
\newcommand{\qeqpostab}[3]{\mathsf{stab}_{#1,#2}(#3)} % iso for stability along #3 of pushout of ##1,2
\newcommand{\qeqex}{\mathsf{ex}}            % exactness iso

\newcommand{\qeqle}{\varepsilon}        % empty list
\newcommand{\qeqlcons}{\cons}        % cons
  % old notation from Maietti-Vickers
%\newcommand{\qeqle}{\mathsf{r}_{0}}        % empty list
%\newcommand{\qeqlsnoc}{\mathsf{r}_{1}}        % snoc
\newcommand{\qeql}{\mathsf{List}}           % list object
\newcommand{\qeqlrec}[3]{\mathsf{rec}^{#1}(#2,#3)} % rec (A,b,g)
\newcommand{\qeqlu}[4]{\mathsf{u}^{#1}_{#2,#3}(#4)}        % list uniqueness (A,b,g,r)





\tikzset{
  no line/.style={draw=none,
    commutative diagrams/every label/.append style={/tikz/auto=false}},
  from/.style args={#1 to #2}{to path={(#1)--(#2)\tikztonodes}}}


%-------------------------------------------from Bartlet 

% Definitions
\newcommand\GCat{\ensuremath{\mathrm{2Cat}_{\mathrm{G}}}}
\newenvironment{tz}[1][]{\begin{aligned}\begin{tikzpicture}[#1]}{\end{tikzpicture}\end{aligned}}



\usepackage[english]{babel}
\usepackage[utf8]{inputenc}
\usepackage{amsmath}
\usepackage{mathtools}
\usepackage{graphicx}
\usepackage[colorinlistoftodos]{todonotes}
\usepackage{xr}
\usepackage{amsthm}
\usepackage{amsmath, graphicx, bbm}
\usepackage[color,matrix,arrow]{xy}
\usepackage{stmaryrd,xspace,amssymb}
\usepackage{graphics,color}
\usepackage[active]{srcltx}
\usepackage{url}
\usepackage{tikz}
\usepackage{tikz-cd}
\usetikzlibrary{calc}
\usepackage{verbatim}
%\usepackage[active,tightpage]{preview}
\usepackage{ragged2e}
\usepackage{epigraph}
\usepackage{xypic}
\xyoption{2cell}
\UseAllTwocells
\usepackage{upgreek}
\usepackage{multicol}
\usepackage{xr-hyper} %The package describes itself as a development version of the standard package xr; it can create hyperlinks to the external documents that are cross-referenced.
\usepackage{hyperref}
\usetikzlibrary{matrix}
\usepackage{media9}
\usepackage{xcolor}
\usepackage{Macros}
\usepackage{Macros2}
\usepackage{diag}
\usepackage{equationalproof}
\usepackage{float}
\usepackage{dirtytalk}
\usepackage{romannum}
\usetikzlibrary{arrows,positioning,automata,shadows,fit,shapes}











